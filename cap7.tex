\chapter{Las Verdaderas Búsquedas}

\section{XENON}

***DONDE ESTÁ
***QUE HACE
***COMO LO HACE
***COMO SE VE
***QUIEN LO MANEJA

The XENON dark matter research project, operated at the Italian Gran Sasso National Laboratory, is a deep underground research facility featuring increasingly ambitious experiments aiming to detect dark matter particles. The experiments aim to detect particles in the form of weakly interacting massive particles (WIMPs) by looking for rare interactions via nuclear recoils in a liquid xenon target chamber. The current detector consists of a dual phase time projection chamber (TPC).

The experiment detects scintillation and ionization produced when particles interact in the liquid xenon volume, to search for an excess of nuclear recoil events over known backgrounds. The detection of such a signal would provide the first direct experimental evidence for dark matter candidate particles. The collaboration is currently led by Italian professor of physics Elena Aprile from Columbia University. 

\section{ICECUBE}

\section{PAMELA}

\section{KAMIOKANDE}

\section{LHC}
    \subsection{ATLAS}
    \subsection{CMS}
 \section{SWGO}
 \section{ALMA}
 \section{PLANCK}
 \section{VERA RUBIN}
 \section{LLP}
 \section{HUBBLE }
 \section{Sucesor del Hubble}
 \section{Appec}
 \section{}