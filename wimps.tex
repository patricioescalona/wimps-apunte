\documentclass[letterpaper,11pt]{report}
\usepackage{float,captdef,multicol}
\usepackage{amssymb,amsmath,amsfonts,fancybox}
\usepackage[activeacute,spanish,es-tabla]{babel}
\usepackage[utf8]{inputenc}
\usepackage[colorlinks]{hyperref}
\usepackage{graphicx}
\usepackage{cancel}
\usepackage{marginnote}

\renewcommand*{\marginnotevadjust}{-0.1cm}
\renewcommand*{\marginfont}{\footnotesize}
\usepackage[right=4.5cm,left=2cm,top=3cm,bottom=3.0cm]{geometry}

%\usepackage[notref]{showkeys} % muestra los labels de las referencias.

\spanishdecimal{.}


\begin{document}

\sffamily

\thispagestyle{empty}
\begin{center}



\vspace{6.5cm}

\rule{15cm}{0.1cm}

\vspace{1.5cm}

{\huge \textsc{\textbf{WIMPS}}}

\vspace{1.5cm}

\rule{15cm}{0.1cm}

\vspace{1.5cm}

Versión del \today

\end{center}

\newpage
\thispagestyle{empty}
\newpage
\setcounter{page}{1}
\pagenumbering{roman}

\pagestyle{plain}
\chapter*{Prefacio}
\addcontentsline{toc}{chapter}{Prefacio}
\bigskip
\bigskip
\bigskip
\bigskip
\bigskip
\bigskip


\emph{Este apunte está siendo escrito por P. Escalona Contreras para registrar su aprendizaje con respecto al \textit{problema de la materia oscura} y la hipòtesis de la existencia de \textit{partìculas débilmente interactuántes} (WIMPs). Esta obra no sería posible sin las invaluables mentes de J. Oliva, A. Zerwekh, B. Diaz, G. Rubilar,  \dots,\dots}

\bigskip


\bigskip

Esta obra ha sido publicada bajo una \href{https://github.com/gfrubi/electrodinamica/blob/master/LICENSE}{licencia GPL v3} INSERTAR LICENCIA. El c'odigo fuente (pdf)\LaTeX, as'i como las figuras en formato editable est'an disponibles en el \href{https://github.com/gfrubi/electrodinamica}{repositorio GitHub }del proyecto CAMBIAR REPOSITORIO y están basados en archivos de G. Rubilar GITHUB DE GRUBI.
\bigskip
\bigskip
\bigskip
\bigskip
\bigskip
\bigskip



\emph{\textquotedblleft The effort to understand the universe is one of the very few things that lifts human life a little above the level of farce, and gives it some of the grace of tragedy."}

\begin{flushright}
Steven Weinberg.
\end{flushright}

\newpage

\tableofcontents
\pagenumbering{arabic}
\setcounter{page}{1}

\chapter{El problema de la masa faltante}
\pagenumbering{arabic}
\setcounter{page}{1}

****HISTORIAqweqsasdas
\chapter{Materia Oscura como Partícula}
** ENFOQUE DE FISICA DE PARTICULAS
**CLASIFICACION DE WIGNER
**PRIMORDIAL BLACK HOLES
**MINIMAL DARK MATTER---->CLASIFICACION DE BIANCHI???
** FROM 't Hooft anomaly matching
** SPIN 1
** MSSM
** ETC 
** ETC 
** ETC

\chapter{Abundancias Cósmicas}
***COSMOLOGIA ESTANDAR**
**primero tres minutos de weinberg**
**Kolb, Turner**
** ETC **

\section{freeze-out}
\section{freeze-in}
\chapter{Detección Directa}
\chapter{Detección Indirecta}

\chapter{Detección en Colisionadores}
\end{document}